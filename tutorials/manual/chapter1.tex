% Copyright (C) 2013 Columbia University in the City of New York and others.
%
% Please see the AUTHORS file in the main source directory for a full list
% of contributors.
%
% This file is part of TerraFERMA.
%
% TerraFERMA is free software: you can redistribute it and/or modify
% it under the terms of the GNU Lesser General Public License as published by
% the Free Software Foundation, either version 3 of the License, or
% (at your option) any later version.
%
% TerraFERMA is distributed in the hope that it will be useful,
% but WITHOUT ANY WARRANTY; without even the implied warranty of
% MERCHANTABILITY or FITNESS FOR A PARTICULAR PURPOSE. See the
% GNU Lesser General Public License for more details.
%
% You should have received a copy of the GNU Lesser General Public License
% along with TerraFERMA. If not, see <http://www.gnu.org/licenses/>.

\chapter{Introduction}

\TF{},  the\emph{Transparent Finite Element Rapid Model Assembler}, is
a software system for the rapid and reproducible construction and exploration of coupled multi-physics models.

TerraFERMA leverages three advanced open-source libraries for
scientific computation that provide high level problem description
(\href{http://fenicsproject.org}{FEniCS}), composable solvers for
coupled multi-physics problems
(\href{https://www.mcs.anl.gov/petsc}{PETSc}) and a science neutral
options handling system
(\href{https://www.imperial.ac.uk/engineering/departments/earth-science/research/research-groups/amcg/software/spud}{SPuD})
that allows the hierarchical management of all model
options. TerraFERMA inherits most of its functionality from the
underlying libraries but adds a layer of control and guidance for
building reusable and reproducible applications.
% A manuscript describing the design and basic functionality of TerraFERMA can be found here.

As most of the language of \TF{} is also inherited from the underlying
libraries it is also well worth looking at the extensive tutorials for
the \href{http://fenicsproject.org/documentation/tutorial}{FEniCS} project which provide an
excellent introduction to both finite elements and UFL (Unified form
language) for describing weak forms.  It is also well worth
understanding the basic PETSc solver objects for nonlinear solvers
(SNES), Linear solvers (KSP) and Preconditioners (PC) as we will use
the same language for describing solver options (although \TF{}
provides some guidance and subsets of options for choosing).

The purpose of these tutorials is to develop hands-on experience with
\TF{} through detailed, step-by-step examples starting with the
simplest problem of Poisson's equation on a unit square to more
complex multi-physics problems such as time-dependent thermal
convection and non-linear magmatic solitary waves.  The key design
principal of \TF{} is that computational models are, at their most
abstract, a \emph{very large} set of scientific and computational
choices and the purpose of \TF{} is to help the user organize, manage
and modify those choices in a manner that is flexible, reproducible
and reusable.

\pagebreak{}
\section{Getting and installing the \TF{} software}
\label{sec:gett-inst-tf}

\TF{} is available as an open-source code under a LGPL 3.0
license.  It is hosted as a
\href{https://github.com/terraferma/terraferma}{git repository}.
Installation information, scripts for building from source and
considerably more information is available on the
\href{https://terraferma.github.io}{Wiki} page. 


%%% Local Variables: 
%%% mode: latex
%%% TeX-master: "tftutorials"
%%% End: 
